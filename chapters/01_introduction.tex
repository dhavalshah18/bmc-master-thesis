\chapter{Introduction}

%\minitoc\pagebreak


\section{Background}
An intracranial aneurysm is a bulge located in a blood vessel in the brain. The rupture of an intracranial aneurysm can is a serious incident that has high fatality and morbidity rates. Unruptured intracranial aneurysms \acr{Unruptured intracranial aneurysms}{UIA} (UIAs) affect approximately $3$-$5 \% $ of the adult population, irrespective of geographical location and/or ethnicity \cite{vlak2011prevalence}. The clinical manifestations of UIAs are subtle, with only approximately $10$-$15\%$ of intracranial aneurysms being symptomatic \cite{friedman2001small}. Therefore, the diagnosis of an intracranial aneurysm is primarily done through the use of imaging modalities such as intra-arterial digital subtraction angiography (IADSA), computed tomography angiography (CTA) and magnetic resonance angiography (MRA). \todo{risk factors?}

\section{Purpose}






