\chapter{Introduction}
\label{chapter1}
%\minitoc\pagebreak
% Aim for 2-3 pages

%\section{Background}
An intracranial aneurysm is a bulge located in a blood vessel in the brain. The rupture of an intracranial aneurysm can lead to a parenchymal hemorrhage, or more commonly to an aneurysmal subarachnoid hemorrhage (aSAH) which is a serious incident that has high case fatality and morbidity rates \cite{Keedy2006}. Unruptured intracranial aneurysms (UIAs) affect approximately $3$-$5 \%$ of the adult population, irrespective of geographical location and/or ethnicity \cite{vlak2011prevalence}. There is also evidence that certain risk groups have an increased risk of aSAH; with approximately 10\% prevalence in individuals with a positive family history for aSAH \cite{bor2014long}. The clinical manifestations of UIAs however, are subtle, with only approximately $10$-$15\%$ of intracranial aneurysms being symptomatic \cite{friedman2001small}. 

It is important that UIAs are detected early during screening to allow for a treatment decision to be made. The diagnosis of an UIA is primarily done through the use of three imaging modalities; intra-arterial digital subtraction angiography (IADSA), computed tomography angiography (CTA) and magnetic resonance angiography (MRA). Previously, IADSA was considered the gold-standard in diagnosis, however MRA does not usually require use of contrast agents, and was preferred for pre-operative cases. More recently, due to the improvements made in the quality of intracranial imaging technologies, the application of CTA and MRA as diagnostic tools has largely increased \cite{Brown2014}. A more detailed comparison and description of these three modalities will be given in Chapter \ref{chapter3}. For management of intracranial aneurysms, the risks of rupture are weighed against the risks associated with intervention, with the three main treatment options being observation, endovascular therapy, and surgical therapy \cite{Keedy2006}. 

%\section{Motivation}
Studies have shown that aneurysms can prove to be quite challenging to detect, with $21$-$40\%$ of all aneurysms being missed during diagnostic screenings as they are dependent on the training and experience of the medical professional. In addition, the detection of an aneurysm is also highly affected by the size and location of the aneurysm \cite{okahara2002diagnostic}. In clinical practice (using TOF-MRA), aneurysm detection is performed by radiologists searching through the axial slices of the scan, and possibly combining that with multi-planar reconstructions or a 3D volume reconstruction before making 2D size measurements of the aneurysm. These 2D manual measures used to assess UIAs lack 3D information and there is substantial inter-observer variability for assessment of aneurysm size or growth \citeauthor{White2001}. 3D measures themselves are time-consuming to carry out, and thus the the clinical workflow could benefit from reliable automatic methods of detection, segmentation and quantification of UIAs from 3D TOF-MRAs. It is essential however that these methods do not negatively impact human observer detection and measurement accuracy. Segmentation methods with high precision, high reproducibility, and low bias in surgical planning have been shown to directly impact results of surgical planning, and automated segmentation of UIAs would enable quantification of aneurysms and may aid the prediction of UIA rupture risk \cite{Taha2015}. For example, shape knowledge of an UIA can be used to assess increase in growth and rupture risk, while quantified shape measurements of UIAs can also be useful for models assessing risks of treatment complications \cite{backes2017elapss, ji2016risk}.

%\section{Goal}
With the necessity of good segmentation of UIAs explained, this thesis aims to analyze and compare the various current available methods employing deep learning. A novel deep learning architecture is also introduced, evaluated and discussed in comparison to current methods, named Triplanar-Net. With the novel method the aim is to focus on reduction of resource usage, and increase in speed, with improvement or little to no effect on accuracy of the resulting segmentation. The novel architecture uses 2D projections to segment 3D volumes, and will be discussed in detail in Chapter \ref{chapter5}. Some previous methods for aneurysm detection and aneurysm segmentation (although scarce) are also discussed in Chapter \ref{chapter2} that also introduce parts methods used with the evaluated networks. The dataset made publicly available as part of the MICCAI2020 ADAM challenge is used to train the discussed networks, and results are shown in \ref{chapter6} along with results on the non-publicly available dataset from the same challenge \cite{Timmins2020}. 

The segmentation results achieved by Triplanar-Net were not on par with the evaluated state-of-the-art (SOTA) method of nnU-Net (which was the winning entry to the challenge), but were better than the results achieved by DeepMedic -- another network for aneurysm segmentation which showed good results on another dataset. Although reduction of resource usage and reduction of inference time was achieved by Triplanar-Net this is not considered an adequate finding to warrant use of this network in an automatic workflow. Therefore, further work is needed to explore the use of 2D projections to segment 3D volumes for the task of aneurysm segmentation in 3D TOF-MRA images.







