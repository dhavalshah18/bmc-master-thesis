\chapter{Introduction}

%\minitoc\pagebreak


\section{Background}
An intracranial aneurysm is a bulge located in a blood vessel in the brain, and the rupture of an intracranial aneurysm is a very serious incident that has high fatality and morbidity rates \todo{add ref + details}. Unruptured intracranial aneurysms (UIAs) affect approximately $3$-$5 \%$ of the adult population, irrespective of geographical location and/or ethnicity \cite{vlak2011prevalence}. The clinical manifestations of UIAs however, are subtle, with only approximately $10$-$15\%$ of intracranial aneurysms being symptomatic \cite{friedman2001small}. \todo{risk of rupture of aneurysm} Therefore, the diagnosis of an intracranial aneurysm is primarily done through the use of imaging modalities such as intra-arterial digital subtraction angiography (IADSA), computed tomography angiography (CTA) and magnetic resonance angiography (MRA). The diagnosis of aneurysm before symptoms arise allows possible intervention, if deemed necessary based on size and location \todo{ref}. \todo{risk factors?} \todo{differences in imaging modalities?}

\section{Motivation}
Due to rapidly growing workload of radiologists and radiology department, it could be beneficial to introduce a reliable method for automated detection of UIAs from diagnostic images of patients.\todo{refs} \todo{more}


\section{Goal}
Design a neural network. Also, focus on trying to reduce need for large computations by attempting to use 2d networks, but still reproduce aneurysm segmentations in 3d. \todo{elaborate and extend}







