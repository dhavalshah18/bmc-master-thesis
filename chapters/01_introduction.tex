\chapter{Introduction}
\label{chapter1}
%\minitoc\pagebreak
% Aim for 2-3 pages

\section{Background}
An intracranial aneurysm is a bulge located in a blood vessel in the brain. The rupture of an intracranial aneurysm can lead to a parenchymal hemorrhage, or more commonly to a subarachnoid hemorrhage which is a very serious incident that has high case fatality and morbidity rates \cite{Keedy2006}. Unruptured intracranial aneurysms (UIAs) affect approximately $3$-$5 \%$ of the adult population, irrespective of geographical location and/or ethnicity \cite{vlak2011prevalence}. The clinical manifestations of UIAs however, are subtle, with only approximately $10$-$15\%$ of intracranial aneurysms being symptomatic \cite{friedman2001small}. 

The diagnosis of an UIA is primarily done through the use of three imaging modalities; intra-arterial digital subtraction angiography (IADSA), computed tomography angiography (CTA) and magnetic resonance angiography (MRA). Previously, IADSA was considered the gold-standard in diagnosis, however MRA does not usually require use of contrast agents, and was preferred for pre-operative cases. More recently, due to the improvements made in the quality of intracranial imaging technologies, the application of CTA and MRA as diagnostic tools has largely increased \cite{Brown2014}. A more thorough comparison and description of these three modalities will be given in Chapter \ref{chapter3}. For management of intracranial aneurysms, the risks of rupture are weighed against the risks associated with intervention, with the three main treatment options being observation, endovascular therapy, and surgical therapy \cite{Keedy2006}. Nonetheless, detection of an aneurysm before it becomes symptomatic allows for the possibility of intervention before rupture occurs.

\section{Motivation}
Studies have shown that aneurysms can prove to be quite challenging to detect, with $21$-$40\%$ of all aneurysms being missed during diagnostic screenings depending on the education and experience of an observer. Detectability of an aneurysm is also highly affected by the size and location of the aneurysm \cite{okahara2002diagnostic}. Coupled with the increasing workload of radiologists and surgeons, the necessity of good image processing steps in medical image analysis becomes much more apparent \todo{maybe ref?}.

Segmentation methods with high precision, high reproducibility, and low bias in surgical planning have been shown to directly impact results \cite{Taha2015} \todo{results of what}. However, segmentation of UIAs remains a challenge in all modalities
\todo{more, necessity of computer aided detection, segmentation in use by radiologists, how modalities are used etc.}


\section{Goal}
With the necessity of good segmentation of UIAs laid out, in this thesis the various methods employing deep learning already available will be analyzed and compared. A novel deep learning architecture is also introduced, evaluated and discussed. With the novel method the aim is to focus on reduction of resource usage, and increase in speed, with improvement or little to no effect on accuracy of the resulting segmentation. The novel architecture uses a combination of 2D and 3D to achieve this, and will be discussed in detail in chapter \todo{add label}. Some deep learning architectures are also introduced in chapter \ref{chapter2} that also introduce parts of the novel method. \todo{I think need to be more concise here, and lay out what exactly was achieved?}







