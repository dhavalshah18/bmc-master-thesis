\chapter{Time-of-Flight Magnetic Resonance Angiography}
\label{chapter3}
% Aim for 5-7 pages
\todo{Check Keedy2006, it gives a good overview of MRA used for diagnosis as well as other modalities}
As discussed previously, Time-of-Flight (TOF) Magnetic Resonance Angiography (MRA) is an imaging modality used for diagnosis of UIAs. The following chapters will discuss deep learning methods to segment UIAs on these images, however it is also valuable to go deeper into the acquisition of these images, as well as discuss and analyse the specific images used.

\section{Dataset}
The dataset used was obtained from the Aneurysm Detection And segMentation (ADAM) Challenge 2020, a medical image analysis challenge organised as part of MICCAI 2020. The train dataset of TOF-MRAs consists of \textbf{113} cases which are split into 93 containing at least one untreated, unruptured aneurysm (35 baseline and 35 follow-up of the same subject, and 23 unique subjects), and 20 scans without intracranial aneurysms. \todo{Add images of each type} \todo{Table?} \todo{Train/Validation split?}

\section{Analysis of dataset}
\todo{Analyse dataset e.g. aneurysm sizes}

\section{Acquisition}



\section{Comparison to other modalities}
\todo{Look for some open source cranial CTA with labels}
\todo{Look for some cranial DSA, maybe ask Supro if you can use the old ones}

