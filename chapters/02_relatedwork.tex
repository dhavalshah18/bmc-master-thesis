\chapter{Related Work}

%\minitoc\pagebreak

\section{Computer Assisted Detection}
\todo{CAD in use in general clinical scenarios}

\todo{CAD use/research with respect to UIAs}

\section{Deep learning based aneurysm detection}
Detection of UIAs using deep learning is an active area of study, whose biggest challenge - as it is for a large number of medical tasks - is obtaining labeled data. A recent challenge \todo{Find journal entry for MICCAI2020 challenge you took part in} allowed multiple teams access to a dataset of UIAs, and produced a variety of approaches tackling the problem. Prior to this, the gold standard seems to have been achieved by Sichermann et. al. \cite{Sichermann2019} with the DeepMedic framework \cite{deepmedic}.
\todo{Gold standard for aneurysm detection - \cite{Sichermann2019}}
\todo{Talk about other methods of TOF-MRA aneurysm detection with DL, e.g. \cite{Ueda2019}}

Deep learning methods have also been used for detection of aneurysms in other parts of the human anatomy, including the aorta... \todo{Talk about various methods related to deep learning in aneurysm detection}

A large number of methods also aim to detect or segment UIAs in computed tomography (CT) angiography images...
\todo{About various image modalities used}

\todo{Differences in domain}

\todo{3d NN's, subsection?}

\todo{2d NN's, subsection?}

\todo{hybrid (3d+2d) NN's, subsection?}




