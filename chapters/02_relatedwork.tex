\chapter{Related Work}
\label{chapter2}
%\minitoc\pagebreak
% Aim for 6-9 pages

Several methods and algorithms have been published that tackle segmentation of UIAs, as well as detection. To easily discuss them they can be separated into non-deep learning and deep-learning based methods. Although in this work the focus is on segmentation of aneurysms, it is also important to analyze methods of detection as some of the work done in that field can also be applied to segmentation. There are also a variety of methods that are not evaluated on MRA images but rather on other modalities, and these are also included.

\section{Non-deep learning-based aneurysm detection and segmentation}
Various methods have been proposed for detection of intracranial aneurysms prior to the surge of deep learning in this field. The methods include some that involve using machine learning methods with handcrafted features, as well as some that involve the use of preprocessing and filtering to detect or segment aneurysms. \citeauthor{Lauric2010} proposed automatic detection of aneurysms in CTA and 3D X-ray rotational angiography (3D RA) that utilizes the Writhe number -- used to measure how much a curve twists and coils -- to characterize surfaces \cite{Lauric2010}. However, the detection algorithm requires a segmented volume of cerebral vasculature, and a large amount of preprocessing. \citeauthor{Yang2011} similarly proposed a method that also requires the segmentation of the cerebral vasculature in the first step \cite{Yang2011}. Following this, the corresponding 3D centerline of the segmentation is computed, and used to determine vessel parts and corresponding endpoints. This serves as an initial sample of aneurysm candidates, which is further extended using thresholding, and finally the sample is reduced using a rule base. \citeauthor{Suniaga2012} in their work aimed to combine all relevant parameters used in these previous studies within one approach \cite{Suniaga2012}. Like \citeauthor{Yang2011}, their method involves computing the 3D centerline from the extracted cerebrovascular segmentation, after which a SVM is used for classification of initial aneurysm candidates based on extracted parameters. Another method proposed by \citeauthor{Hentschke2014} is one that aims to classify multimodal angiographic images (in CTA, 3D-RA, and MRA) without the required first step of segmentation that all the other previous methods employ \cite{Hentschke2014}. Following some preprocessing, Volumes of Interest (VOI) are found using connected-component analysis (CCA), and low-level and high-level features are computed on each VOI. To reduce false-positive rate, their method proposes using two variants for classification; a linear and a non-linear classification variant. The variants comprise a rule-based system followed by a linear determinant function, and either SVM, an alternate decision tree, or a LogitBoost boosting method respectively. 

The performance of these detection methods are measured using two evaluation metrics; sensitivity and false-positive rate. Sensitivity being the measure of the proportion of positives that are correctly identified, and false-positive rate being a measure of the proportion of diagnoses incorrectly classified as positive. It is possible to compare the algorithms proposed by \citeauthor{Hentschke2014}, \citeauthor{Suniaga2012}, and \citeauthor{Yang2011} -- as they use MRA images, but results reported by \citeauthor{Lauric2010} can only be compared to those reported by \citeauthor{Hentschke2014}. Nonetheless, outright comparison between the methods is still difficult. The sensitivity of the detection methods for TOF-MRA were reported above $90\%$ for all three of these studies, however \citeauthor{Hentschke2014} achieved a better sensitivity for smaller aneurysms ($< 5$ mm) of $93\%$. They were also able to detect different kinds of aneurysms (sacular and\citeauthor{Lauric2010} reported $100\%$ sensitivity compared to $95\%$ sensitivity reported by \citeauthor{Hentschke2014}. Another issue with the proposed method by \citeauthor{Hentschke2014} is the high false-positive rate in comparison with all the other studies. However, their method does not require any manual preprocessing unlike all the other proposed methods.

Non-deep learning segmentation methods have also been proposed, and they include some that are fully automatic as well as semi-automatic methods that require some form of input from the user. A semi-automatic segmentation method was proposed by \citeauthor{Firouzian2011} which uses Geodesic Active Contours (GAC) for detection of aneurysms in CTA \cite{Firouzian2011}.Their proposed method is implemented in the level set framework, in which a surface is evolved to capture the aneurysm. The evolution is steered by image intensity statistics defined around the single user-defined seed point. \citeauthor{Bogunovic2011} introduced an automatic multimodal segmentation method for aneurysms (as well as cerebral vasculature) based on Geodesic Active Regions (GAR) \cite{Bogunovic2011}. This method is similar to that of \citeauthor{Firouzian2011} in that it also uses a deformable model within the level set framework, however the method also combines region-based descriptors with gradient ones to drive the evolution toward vascular boundaries. Being an automatic segmentation method, it also does not require the manual definition of a seed point. However, their method is focused toward cerebrovascular segmentation, with the added bonus of being able to segment aneurysms and therefore is not a dedicated solution for the problem of aneurysm segmentation \todo{Explain/read more why can't compare with this method}. In a work published by \citeauthor{Sen2013}, two other aneurysm segmentation methods were evaluated and another method -- Threshold-Based Level Set (TLS) was proposed \cite{Sen2013}. TLS combines GAC and the Chan-Vese model \cite{Chan2001} within the level set framework. Their method also integrates both region and boundary information, and makes use of a global threshold and gradient magnitude to form the function that evolves the segmentation towards the cerebral aneurysm. The Chan-Vese model is used to calculate the initial threshold value, after which it is iteratively updated throughout the segmentation process. 

The norm for evaluation of segmentation methods for recent publishings involves the use of metrics such as Dice Similarity Coefficient, Hausdorff Distance, and Intersection-over-Union for example. In the studies published by \citeauthor{Firouzian2011}, \citeauthor{Bogunovic2011} and \citeauthor{Sen2013} these metrics were not reported making it difficult to compare the methods, aside from the obvious issue of comparing methods targeting different modalities. From these methods, the one proposed by \citeauthor{Sen2013} is geared towards solving the problem of aneurysm segmentation -- in CTA images and not MRA images --  and also does not require the manual setting of a seed point or intensity threshold. It aims to combine the method described by \citeauthor{Firouzian2011} with another method, and thus could be assumed to be more appropriate for the problem. \todo{This is bullshit mostly}


\todo{Explain why these are interesting, what to take away from these methods, if ML methods are better than more basic ones and why/why not, and more on results.} 

\section{Deep learning-based aneurysm detection and segmentation}
Non-deep learning methods require handcrafted engineered features that are defined in terms of mathematical equations. These features are then used as inputs to models that are trained to classify, detect or segment. In comparison, deep learning algorithms can automatically learn feature representations from data, and has thus seen increasing usage in radiology \cite{Hosny2018}. Specifically in the field of intracranial aneurysm management, deep learning methods have also shown to rapidly be becoming a promising aid; not only in detection of UIAs, but also in evaluating rupture risks, and predicting treatment outcomes \cite{Shi2020}.

\todo{Go into detection}
\todo{First talk \cite{Ueda2019} which is 2D detection in MRA, then talk \cite{Yang2021} which is 3D detection in CTA, and then talk \cite{Sichermann2019} + \cite{Faron2020} which is like similar authors and same architecture (for 3D MRA detection) I think but just both architecture and then evaluation. Could/should also talk about ADAM challenge, and other MICCAI 2020 challenge for aneurysms in CTA if you can find published papers.}
\citeauthor{Ueda2019} propose to use a standard ResNet-18 architecture to detect intracranial aneurysms from TOF-MRA images \cite{He2016, Ueda2019}. Firstly training candidates are extracted by detecting arterial abnormalities from cerebral arterial curvatures. Then patches are extracted from the training data set and the network is trained with the aneurysm annotations. The study involved the use of a large amount of data acquired from four medical institutions, however all TOF-MRA images were positive (i.e. all images used contained one or more aneurysm(s)). Furthermore, the method proposed uses 2D axial images rather than full 3D volumes. 

\todo{Go into segmentation}
\todo{Could mention that this seems like more of a niche problem that detection, especially with deep learning methods.}

\todo{Discuss some pros and cons of introducing deep learning to this, maybe from \cite{Kallmes2021}}

\todo{Differences in domain}

\todo{3d NN's, subsection?}

\todo{hybrid (3d+2d) NN's, subsection?}

\todo{CAD, AI in intracranial aneurysm diagnosis from \cite{Shi2020}}

\todo{Table with all methods discussed with results}






