\chapter{Related Work}

%\minitoc\pagebreak
% Aim for 6-9 pages

\section{Aneurysm detection}
\subsection{Non-deep learning-based aneurysm detection}
Various methods have been proposed for detection of intracranial aneurysms prior to the surge of deep learning in this field. A bulk of these methods are not fully automatic, i.e. they require an initial input from the user -- such as a seed point. Firouzian et. al. proposed a method using Geodesic Active Contours for detection of aneurysms in CTA \cite{Firouzian2011}. The framework involves evolving an embedding function towards the boundaries of the aneurysm in 3D by using image intensity, gradient magnitude and intensity variance. \todo{Explain why this is interesting, why this method over others, talk results} 
Another method proposed by Hentschke et. al. involves the use of non-linear classification algorithms -- support-vector-machine, alternate decision-tree, and LogitBoost, using a feature vector $F$ computed based on certain key characteristics of aneurysms \cite{Hentschke2014}. Their method is aimed at detecting aneurysms in both CTA and MRA volumes. Their method specifically performs better on the different types of aneurysms, despite the fact that other existing methods achieved better performance than theirs on single modalities. Another major advantage their method proposed is the lack of required (manual) preprocessing, which the other existing methods required. \todo{Talk about what to take away from these methods, i.e. how more recent methods are better, or what parts of these methods I can use}

\subsection{Deep learning-based aneurysm detection}
Detection of UIAs using deep learning is an active area of study, whose biggest challenge - as it is for a large number of medical tasks - is obtaining labeled data. A recent challenge \todo{Find journal entry for MICCAI2020 challenge you took part in} allowed multiple teams access to a dataset of UIAs, and produced a variety of approaches tackling the problem. Prior to this, the gold standard seems to have been achieved by Sichermann et. al. \cite{Sichermann2019} with the DeepMedic framework \cite{deepmedic}.
\todo{Gold standard for aneurysm detection - \cite{Sichermann2019}}
\todo{Talk about other methods of TOF-MRA aneurysm detection with DL, e.g. \cite{Ueda2019}}

Deep learning methods have also been used for detection of aneurysms in other parts of the human anatomy, including the aorta... \todo{Talk about various methods related to deep learning in aneurysm detection}

A large number of methods also aim to detect or segment UIAs in computed tomography (CT) angiography images \todo{About various image modalities used}

\todo{Differences in domain}

\todo{3d NN's, subsection?}
Neural networks that use 3D convolutional kernels are referred to simply as 3D neural networks. 


\todo{hybrid (3d+2d) NN's, subsection?}

%\section{Computer Assisted Detection}
%\todo{CAD in use in general clinical scenarios}
%
%\todo{CAD use/research with respect to UIAs}
%
%\todo{AI in intracranial aneurysm diagnosis from \cite{Shi2020}}

\section{Aneurysm segmentation}





