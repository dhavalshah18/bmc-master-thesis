\chapter{Related Work}
\label{chapter2}
%\minitoc\pagebreak
% Aim for 6-9 pages

Several methods and algorithms have been published that tackle segmentation of UIAs, as well as detection. To easily discuss them they can be separated into non-deep learning and deep-learning based methods. Although in this work the focus is on segmentation of aneurysms, it is also important to analyze methods of detection as some of the work done in that field can also be applied to segmentation. There are also a variety of methods that are not evaluated on MRA images but rather on other modalities, and these are also included.

\section{Non-deep learning-based aneurysm detection and segmentation}
Various methods have been proposed for detection of intracranial aneurysms prior to the surge of deep learning in this field. The methods include some that involve using machine learning methods with handcrafted features, as well as some that involve the use of preprocessing and filtering to detect or segment aneurysms. \citeauthor{Lauric2010} proposed automatic detection of aneurysms in CTA and 3D X-ray rotational angiography (3D RA) that utilizes the Writhe number -- used to measure how much a curve twists and coils -- to characterize surfaces \cite{Lauric2010}. However, the detection algorithm requires a segmented volume of cerebral vasculature, and a large amount of preprocessing. \citeauthor{Yang2011} similarly proposed a method that also requires the segmentation of the cerebral vasculature in the first step \cite{Yang2011}. Following this, the corresponding 3D centerline of the segmentation is computed, and used to determine vessel parts and corresponding endpoints. This serves as an initial sample of aneurysm candidates, which is further extended using thresholding, and finally the sample is reduced using a rule base. \citeauthor{Suniaga2012} in their work aimed to combine all relevant parameters used in these previous studies within one approach \cite{Suniaga2012}. Like \citeauthor{Yang2011}, their method involves computing the 3D centerline from the extracted cerebrovascular segmentation, after which a Support Vector Machine (SVM) is used for classification of initial aneurysm candidates based on extracted parameters. Another method proposed by \citeauthor{Hentschke2014} is one that aims to classify multimodal angiographic images (in CTA, 3D-RA, and MRA) without the required first step of segmentation that all the other previous methods employ \cite{Hentschke2014}. Following some preprocessing, Volumes of Interest (VOI) are found using connected-component analysis (CCA), and low-level and high-level features are computed on each VOI. To reduce false-positive rate, their method proposes using two variants for classification; a linear and a non-linear classification variant. The variants comprise a rule-based system followed by a linear determinant function, and either SVM, an alternate decision tree, or a LogitBoost boosting method respectively. 

The performance of these detection methods are measured using two evaluation metrics; sensitivity and false-positive rate. Sensitivity being the measure of the proportion of positives that are correctly identified, and false-positive rate being a measure of the proportion of diagnoses incorrectly classified as positive. It is possible to compare the algorithms proposed by \citeauthor{Hentschke2014}, \citeauthor{Suniaga2012}, and \citeauthor{Yang2011} -- as they use MRA images, but results reported by \citeauthor{Lauric2010} can only be compared to those reported by \citeauthor{Hentschke2014}. Nonetheless, outright comparison between the methods is still difficult. The sensitivity of the detection methods for TOF-MRA were reported above $90\%$ for all three of these studies, however \citeauthor{Hentschke2014} achieved a better sensitivity for smaller aneurysms ($< 5$ mm) of $93\%$. They were also able to detect different kinds of aneurysms (sacular and\citeauthor{Lauric2010} reported $100\%$ sensitivity compared to $95\%$ sensitivity reported by \citeauthor{Hentschke2014}. Another issue with the proposed method by \citeauthor{Hentschke2014} is the high false-positive rate in comparison with all the other studies. However, their method does not require any manual preprocessing unlike all the other proposed methods.

Non-deep learning segmentation methods have also been proposed, and they include some that are fully automatic as well as semi-automatic methods that require some form of input from the user. A semi-automatic segmentation method was proposed by \citeauthor{Firouzian2011} which uses Geodesic Active Contours (GAC) for detection of aneurysms in CTA \cite{Firouzian2011}.Their proposed method is implemented in the level set framework, in which a surface is evolved to capture the aneurysm. The evolution is steered by image intensity statistics defined around the single user-defined seed point. \citeauthor{Bogunovic2011} introduced an automatic multimodal segmentation method for aneurysms (as well as cerebral vasculature) based on Geodesic Active Regions (GAR) \cite{Bogunovic2011}. This method is similar to that of \citeauthor{Firouzian2011} in that it also uses a deformable model within the level set framework, however the method also combines region-based descriptors with gradient ones to drive the evolution toward vascular boundaries. Being an automatic segmentation method, it also does not require the manual definition of a seed point. However, their method is focused towards cerebrovascular segmentation, with the added bonus of being able to segment aneurysms and therefore is not a dedicated solution for the problem of aneurysm segmentation, thus the results obtained for aneurysm segmentation are not a priority in the work. In a work published by \citeauthor{Sen2013}, two other aneurysm segmentation methods were evaluated and another method -- Threshold-Based Level Set (TLS) was proposed \cite{Sen2013}. TLS combines GAC and the Chan-Vese model \cite{Chan2001} within the level set framework. Their method also integrates both region and boundary information, and makes use of a global threshold and gradient magnitude to form the function that evolves the segmentation towards the cerebral aneurysm. The Chan-Vese model is used to calculate the initial threshold value, after which it is iteratively updated throughout the segmentation process. 

The norm for evaluation of segmentation methods for recent publications involves the use of metrics such as Dice Similarity Coefficient (DSC), Hausdorff Distance (HD), and Intersection-over-Union for example. In the studies published by \citeauthor{Firouzian2011}, \citeauthor{Bogunovic2011} and \citeauthor{Sen2013} these metrics were not reported making it difficult to compare the methods, aside from the obvious issue of comparing methods targeting different modalities. From these methods, the one proposed by \citeauthor{Sen2013} is geared towards solving the problem of aneurysm segmentation -- in CTA images and not MRA images -- and also does not require the manual setting of a seed point or intensity threshold. It aims to combine the method described by \citeauthor{Firouzian2011} with another method, and thus could be assumed to be more appropriate for the problem. 

Although the introduction of deep learning-based methods has vastly increased performance of automatic aneurysm segmentation methods, the machine learning methods proposed prior were a valuable first step to solving the problem. 

\section{Deep learning-based aneurysm detection and segmentation}
Non-deep learning methods require handcrafted engineered features that are defined in terms of mathematical equations. These features are then used as inputs to models that are trained to classify, detect or segment. In comparison, deep learning algorithms can automatically learn feature representations from data, and has thus seen increasing usage in radiology \cite{Hosny2018}. Specifically in the field of intracranial aneurysm management, deep learning methods have also shown to rapidly be becoming a promising aid; not only in detection of UIAs, but also in evaluating rupture risks, and predicting treatment outcomes \cite{Shi2020}.

\citeauthor{Ueda2019} propose to use a standard ResNet-18 architecture to detect intracranial aneurysms from TOF-MRA images \cite{He2016, Ueda2019}. Firstly, training candidates are extracted by detecting arterial abnormalities from cerebral arterial curvatures. Then patches are extracted from the training data set and the network is trained with the aneurysm annotations. The study involved the use of a large amount of data acquired from four medical institutions, however all TOF-MRA images were positive (i.e. all images contained one or more aneurysm(s)). Furthermore, the method proposed uses 2D axial images rather than full 3D volumes. An algorithm proposed by \citeauthor{Joo2020} extends the one introduced by \citeauthor{Ueda2019} by detecting intracranial aneurysms in 3D TOF-MRA images using a 3D ResNet architecture \cite{He2016, Joo2020}. \todo{Pros + cons of \cite{Joo2020}}. An interesting, novel method of detection was proposed by \citeauthor{Nakao2018} for 3D MRA; a voxel-based Convolutional Neural Network (CNN) classifier is used, the inputs to which are 2D images generated from VOIs of MRA images by applying an MIP algorithm \cite{Nakao2018}. \todo{More details, pros + cons}. \todo{Talk about recent ADAM challenge} \todo{Add \cite{Duan2020}}

Again, directly comparing results between studies is difficult because of the different datasets and evaluation criteria. \citeauthor{Nakao2018} achieved $94.2\%$ sensitivity with $2.6$ FPs/case, however no external test set was included and the specificity (proportion of negatives correctly identified) was not reported. The dataset used in the study for evaluation also only included positive cases. Comparatively, \citeauthor{Joo2020} achieved high sensitivity and specificity, and compared to other studies theirs was one of the few reporting a higher specificity compared to human performance ($98\%$ to $89\%$). The datasets used for evaluation also included examinations without aneurysms unlike \citeauthor{Ueda2019} and \citeauthor{Nakao2018}, and an external test set was also included in the evaluation to further demonstrate the effectiveness and robustness of their algorithm.

Segmentation of UIAs is a difficult problem as UIAs occur at various locations relative to vessels. Therefore, studies describing methods of segmenting intracranial aneurysms using deep learning are not as abundant as for detection, despite segmentation being an important problem. \citeauthor{Park2019} propose the HeadXNet CNN for segmentation of intracranial aneurysms from CTA volumes \cite{Park2019}. Their method incorporates an encoder-decoder structure, much like the 3D U-net architecture -- known for being a standard model used for segmentation for a variety of use cases \cite{3dunet}. \todo{pros + cons}. \citeauthor{Liu2021} used a modified U-net architecture with dense blocks (connects each layer to every other layer, allowing an identity transform in between individual layers) \cite{Liu2021}. This study focused on segmenting intracranial aneurysms in 3D rotational DSA, but it still goes to show the versatility and adaptability  of the 3D U-net architecture. A surface-based deep learning framework for segmenting intracranial aneurysms in TOF-MRA has also been proposed; unlike usual volume-based pipelines, this method performs segmentation on samples of a cerebrovascular surface representation \cite{Yang2020}. However, this framework first requires to semi-automatically obtain the surface models of the principal brain arteries. \citeauthor{Sichermann2019} propose a method for detection and segmentation of intracranial aneurysms from 3D TOF-MRA data based on an open-source neural network --  DeepMedic \cite{Sichermann2019}. The DeepMedic framework is a CNN for voxelwise classification of medical imaging data after training with 3D patches at multiple scales. Their architecture contains 2 pathways; the pathways are identical apart from the input of the second pathway being a subsampled version of the first. Their study uses a large dataset, however -- like other studies -- they only use volumes that contain aneurysms. \todo{Find ADAM2020}

The results of methods used for segmentation with deep learning methods are the most important for comparisons with the proposed architecture in chapter \ref{chapter5}. Comparing results of different modalities would not be as important as for the same modality; so the studies by \citeauthor{Yang2020}, \citeauthor{Sichermann2019} and \todo{ADAM2020} can be compared. Although not the same datasets are used in each case, \citeauthor{Sichermann2019} produce a maximum Dice Similarity Coefficient of $0.53 \pm 0.30$, \citeauthor{Yang2020} obtain a maximum Dice Similarity Coefficient of $0.76 \pm 0.27$ and \todo{ADAM2020} obtain a maximum score of $0.41$. \todo{Reason why scores are so different, and which are more important}


\todo{Discuss some pros and cons of introducing deep learning to this, maybe from \cite{Kallmes2021}}

\todo{Differences in domain}

\todo{hybrid (3d+2d) NN's, subsection?}

\todo{CAD, AI in intracranial aneurysm diagnosis from \cite{Shi2020}}

\todo{Add table with all results}

\todo{Maybe go into inspiration for my architecture? But its weird to do that because I haven't introduced my architecture yet so how could I talk about the different parts of it and the related works to that?}

\begin{table}[t]
	\centering
	\begin{tabular}{l c r r r}
		Task & Modality & Sensitivity & False positive rate & DSC \\
		\hline
		Detection & & & & \\
		\hspace{5mm}\citeauthor{Hentschke2014} & CTA, 3D RA, MRA & & & \\
		\hspace{5mm}\citeauthor{Joo2020}* & TOF-MRA & & & \\
		\hspace{5mm}\citeauthor{Lauric2010} & CTA, 3D RA & & & \\
		\hspace{5mm}\citeauthor{Nakao2018}* & MRA & & & \\
		\hspace{5mm}\citeauthor{Suniaga2012} & MRA & & & \\
		\hspace{5mm}\citeauthor{Ueda2019}* & TOF-MRA (2D) & & & \\
		\hspace{5mm}\citeauthor{Yang2011} & MRA & & & \\		
		& & & & \\
		Segmentation & & & & \\
		\hspace{5mm}\citeauthor{Bogunovic2011} & CTA & & & \\
		\hspace{5mm}\citeauthor{Firouzian2011} & CTA & & & \\
		\hspace{5mm}\citeauthor{Liu2021}*  & 3D Rotational DSA & & & \\
		\hspace{5mm}\citeauthor{Park2019}* & CTA & & & \\
		\hspace{5mm}\citeauthor{Sen2013} & CTA & & & \\
		\hspace{5mm}\citeauthor{Sichermann2019}* & TOF-MRA & & & \\
		\hspace{5mm}\citeauthor{Yang2020}* & TOF-MRA & & & \\

		
	\end{tabular}
	\caption{\todo{Caption} \todo{Mark DL methods somehow in table}}
	\label{table:related_work}
\end{table}







