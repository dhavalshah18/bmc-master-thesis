\PassOptionsToPackage{table,svgnames,dvipsnames}{xcolor}

\usepackage[utf8]{inputenc}
\usepackage[T1]{fontenc}
\usepackage[sc]{mathpazo}
\usepackage[american]{babel}
\usepackage[autostyle]{csquotes}
\usepackage[%
  backend=biber,
  bibwarn,
  url=false,
  style=numeric,
  maxnames=4,
  maxbibnames=99,
  maxcitenames=1,
  firstinits,
  uniquename=init,
  abbreviate=false,
  doi=false,
  isbn=false]{biblatex}
\usepackage{listings}
\usepackage{lstautogobble}
\usepackage{booktabs}
\usepackage[final]{microtype}
\usepackage[toc,nonumberlist,acronym]{glossaries}
\renewcommand*{\glspostdescription}{} %Den Punkt am Ende jeder Beschreibung deaktivieren

\usepackage{amsmath,amssymb,marvosym} % Math

\usepackage{lipsum} % Lorem ipsum. You can safely remove this when you entered your content!

%\usepackage{fancyhdr}

%\usepackage{xspace} % Für manuelle Abstände

% PDF
\usepackage[pdfauthor={YOUR NAME},
	pdftitle={YOUR TITLE},
	baseurl={http://campar.in.tum.de},
	breaklinks,hidelinks]{hyperref}

\usepackage{url}
%\usepackage{pdflscape} % einzelne Seiten drehen können

% Tabellen
%\usepackage{multirow} % Tabellen-Zellen über mehrere Zeilen
%\usepackage{multicol} % mehre Spalten auf eine Seite
%\usepackage{tabularx} % Für Tabellen mit vorgegeben Größen
%\usepackage{longtable} % Tabellen über mehrere Seiten
%\usepackage{array}
%\usepackage{float}
%\usepackage{rotating} % Tabellen im Querformat

% Bilder
\usepackage{graphicx} % Bilder
\usepackage{color} % Farben
%\usepackage{floatflt} % Textumfluss
\usepackage{caption} % Verbesserte Untertitel
\usepackage{subfigure} % mehrere Abbildungen nebeneinander/übereinander
%\newcommand{\subfigureautorefname}{\figurename} % um \autoref auch für subfigures benutzen
\usepackage[all]{hypcap} % Beim Klicken auf Links zum Bild und nicht zu Caption gehen
%\usepackage[section]{placeins} % Bilder nur in zugehöriger Section unterbringen

\usepackage{scrhack} % necessary for listings package

\usepackage[nohints]{minitoc} % Table of content at each chapter
\usepackage{ifthen}
\usepackage{enumitem}