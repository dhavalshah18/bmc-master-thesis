% Basic information for cover & title page
\newcommand*{\getUniversity}{Technische Universität München}
\newcommand*{\getFaculty}{Department of Informatics}
\newcommand*{\getTitle}{Segmenting 3D intracranial aneurysms in Time-of-Flight Magnetic Resonance Images.}
\newcommand*{\getTitleGer}{Segmentierung von intrakraniellen 3D-Aneurysmen in Flugzeit-Magnetresonanzbildern.}
\newcommand*{\getAuthor}{Dhaval Shah}
\newcommand*{\getDoctype}{Master's Thesis in Biomedical Computing}
\newcommand*{\getSupervisor}{Prof.~Dr.~Bjoern~Menze}
\newcommand*{\getAdvisor}{Suprosanna Shit}
\newcommand*{\getSubmissionDate}{\today}
\newcommand*{\getSubmissionLocation}{München}
	
% #################
% ### SHORTCUTS ###
% #################

\newcommand{\TUM}{Technische Universit\"at M\"unchen}

\newcommand{\bigLineSpacing}[1]{
	\ifthenelse{\equal{#1}{ON}}{
		% Flexible whitespace between paragraphs
		\setlength{\parskip}{3mm plus4mm minus2mm}
		\linespread{1.1}
	}{
		% Whitespace OFF
		\setlength{\parskip}{0mm}
		\linespread{1.0}
	}
}

% Todos in text
\newcommand{\todo}[1]{
	{\color{red} TODO: #1} \normalfont
}
\newcommand{\info}[1]{
	{\colorbox{blue}{\color{white}(INFO: #1)}}
}
\newcommand{\delete}[1]{
	{\color{blue} DELETE?: #1} \normalfont
}

% Easy acronym
\newcommand{\acr}[2] {
	\newacronym{#1}{#1}{#2}
}

% Fixed-width image
\newcommand{\img}[4]{
	\begin{figure}[!hbt]
		\centering
		\vspace{1ex}
		\includegraphics[width=#2]{figures/#1}
		\caption[#4]{#3}
		\label{fig:#1}
		\vspace{1ex}
	\end{figure}
}
% Image with file width
\newcommand{\imgfile}[3]{
	\begin{figure}[!hbt]
		\centering
		\vspace{1ex}
		\includegraphics{figures/#1}
		\caption[#3]{#2}
		\label{fig:#1}
		\vspace{1ex}
	\end{figure}
}
% Bild todo
\newcommand{\todoimg}[2]{
	\begin{figure}[!hbt]
		\centering
		\vspace{2ex}
		\includegraphics[width=6cm]{settings/todo}
		\caption{\todo{#2}}
		\label{fig:#1}
		\vspace{2ex}
	\end{figure}
}
% Image on the right side
\newcommand{\imgright}[4]
{
	\begin{floatingfigure}[r]{#2}
		%\centering
		\includegraphics[width=#2]{figures/#1}
		\captionsetup{width=#2}
		\caption[#4]{#3}		
		\label{fig:#1}
	\end{floatingfigure}
}